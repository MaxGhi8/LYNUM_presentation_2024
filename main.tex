\documentclass{beamer}
\usepackage{Config/style/uniReg}

% Use the theme color best suited for your departement. 
% Available themes (case sensitive) : Red, Yellow, GreenBrown, DarkBlue, Black&White (default)
\themecolor{Red}

% Change visibility of the logo in the background. Value can be 0, 1 or 2
\visibilitySiegel{0}

% title and subtitle
\title{Operator learning for multi-patch domains}
\subtitle{}

% For multiple authors/presenters :
%   1. Comment \singleAuthor and use \multAuthors as in example
%   2. a- For same faculty/institute, you can use \AuthorInstitute{...}
%         instead if of \inst{...}
%      b- For different institutes, comment the line \AuthorInstitute{...} and use
%         \inst{...} together with \institute{...}
\singleAuthor{Massimiliano Ghiotto}
\AuthorInstitute{University of Pavia}
%\multAuthors{Antoine Gansel\inst{1} \and Jullie Cailler\inst{2}}

% \Collaborators{{Julie Cailler\inst{2}}} 
\Supervisors{Carlo Marcati\inst{1} \and {Giancarlo Sangalli\inst{1}}}
\institute{{\inst{1}University of Pavia}}

\begin{document}

\frame{\titlepage}

\addtocounter{framenumber}{-1}
\setbeamertemplate{footline}[footline-body]
\setbeamertemplate{background}[background-body]

\section{Fourier Neural Operator}

\begin{frame}{title}
    content
\end{frame}

\backmatter
\end{document}


To begin a  new line with some fixed vertical space
\vspace{\baselineskip}


\begin{block}{title}
\end{block}


% \begin{verbatim}  This begins a verbatim environment where the text is displayed exactly as it is written. This is useful for showing code snippets


% \begin{itemize}[<+->]  This starts an itemized (bulleted) list. The [<+->] option specifies that each item in the list should be revealed one by one as the presentation progresses (incrementally).


% \begin{frame}[fragile]{Writing a Simple Slide}  This starts a new slide (frame) titled "Writing a Simple Slide". The fragile option is used because the frame will contain verbatim text, which needs special handling in Beamer.


\uncover<2->{ <content> }  this means that the `content` is veasible from the second transition of the slide
The \uncover command makes the text visible starting from the specified slide number.


\begin{columns} % adding [onlytextwidth] the left margins will be set correctly
    \begin{column}{0.50\textwidth}
        
    \end{column}%
    \begin{column}{0.50\textwidth}
        
    \end{column}
\end{columns}


\textcolor{<color name>}{text}, \emph{}, \alert{} --> commands to brings the attention somewhere


% for a single monocrome block
\begin{themedColorBlock}{ `title` }
    `content`
\end{themedColorBlock}


% for block with highlighted title
\begin{themedTitleBlock}{ `title` }
    `content`
\end{themedTitleBlock}


% implemented styles for block's color
\themecolor{Black&White}, \themecolor{Red}, \themecolor{Yellow}, \themecolor{GreeenBrown}, 
\themecolor{DarkBlue}, \themecolor{Orange}, \themecolor{BlueGreen}, \themecolor{Green}, \themecolor{Blue}